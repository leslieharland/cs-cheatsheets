\documentclass[11pt]{article}
\usepackage{listings}
\begin{document}
\noindent
\section*{L3 notes}
\subsection*{Some terms}
d(h * f) h convolve with f \\
Know what the math equations are doing and apply them\\
%There is no blue cell after green cell
Pad the sides with 0 and shift the kernel across the numbers\\
floor(kernel size / 2)\\\\
$[1,1,1 ]$ symmetric\\
Linear (shift invariant) = output Is scaled accordingly\\
\\
\
Correlation filter to shift pixels in image
\\\\
Normalised = summed up add to 1\\
Effects are negligible for large images\\\\
Gaussian kernel depends more on the sigma\\\\
Salt and pepper = isolated pixels that are problematic\\
Smooth away the impulse and spread it all around\\
Impulse is salt without the pepper\\\\
Higher signal amplify both the signals and noise
\\\\
Histogram stretching adjust contrast\\\\
\
Bitmap save the pixels independently\\
\subsection*{Laplacian}
the process to get the residual is [upsample (fill with 0s) $\rightarrow$ blur $\rightarrow$ subtract from original]\\
(the reason for the blur is to "spread out" the values over the filled-in black pixels)
\newpage
\noindent
Q: Does spatial quantization mean the quantisation of values of x and y? i.e. x and y can only take discrete integer values?\\
A: spatial quantization basically happens because we cant represent continuous x and y coordinates in a discrete system\\
\\
Q: It seems that you always regard intensity as integers, is it a routine for this course?\\
A: Yes and no\\
\\
Q: does adjusting abstract and brightness simply refer to the linear mapping or also include the non-linear gamma mapping\\
A: Non-linear mapping also just the contrast\\
Photography operation we don’t use stretching ( we use images that stretch over the entire range)
\\\\
Q: In practice, is convolution used more than correlation?\\
A: Convolution\\
\\
Q: Is reconstructing the original image using laplacian pyramid 100\% lossless?\\
A: Up to the difference of quantisation
\\\\
Q: Can we compensate motion blur by finding the motion blur kernel and use the inverse kernel to demotion blur?\\
A: Yes, finding the kernel is always hard
\\\\
Q: Is convolution invertible? In a sense that let's say we have a “convolved” image, is there always a filter such that we can recover the original image using convolution? e,g, undoing blur convolution using a specific sharpening convolution\\
A: If we throw away half of the pixels
\\\\
Q: Does laplacian pyramids take 2* space than gaussian pyramids simply because laplacian pyramids have 2 sets of the pyramids (gaussian + residual)?\\
A: Same amount of space
\\\\
%Low pass filtering signal into low domain frequency
%High frequency represent sharp edges
%Slow changing parts to remain and we take care of the fast changing pixels
%We can take it as a blurring filter




%2k + 1 is the size of the kernel

\section*{L4 notes}
\subsection*{Some terms}
mean filter does not remove all noise and blurs the image
only isolated values 0-255
\\\\
median filter remove all noise and blurs the image slightly
arrange in increasing starting from 0
\\\\
difference filter sum up all the values in row 
same value for the entire row
\\\\
each pixel has a tangent plane
\\\\
kernel size controls the strength of the filter
\\\\
border problem output is reduced in size for the pixels along the edge
\\\\
 the image patch and the template always contain positive numbers, cos $\Theta \in [0, 1]$, i.e., the output of normalized cross-correlation is normalized to the interval [0,1], where 0 means no similarity and 1 means a complete similarity. 
\\\\
region of interest can benefit template matching
\\\\
image has discrete representation we need approximation, positive gradient value when the image change from dark to bright and negative when reversed image
\\\\
Image sharpening g(x,y)=f(x,y)- f(x,y) $\circ $ h(x,y)
\\\\
Magnitude = $\sqrt{(gx2 + gy2 )}$\\\\
Approximated magnitude = $|$gx $|$ + $|$gy $|
$
\\
non maxim suppression
if edge has a magnitude too small connected to another pixel above a threshold, don’t prune 
\\\\
extract edges to detect start and end edges , useful for 3d to know the shape and geometry\\
why? resilient and lighting and color useful for recognition
\\
\\
\subsection*{Template matching }
This object is now the template (kernel) and by correlating an image with this template, the output image indicates where the object is. Each pixel in the output image now holds a value, which states the similarity between the template and an image patch (with the same size as the template) centered at this particular pixel position. The brighter a value, the higher the similarity.
\\\\
Purely correlation since strongest response is when original image exactly matches output\\
what happens when u zoom the baby's face 2x (refer to image)?
\\\\
Template is not symmetrical
\\
\subsection*{Neighborhood processing }\\
Neighbor pixels play a role when determining the output value of a pixel 
\\
\subsection*{Convolution vs Correlation }\\
Convolution (rotated 180 degrees)\\
h(i, j ) $\cdot$ f (x - i, y - j ) 
\\\\
Correlation’\\
h(i, j ) $\cdot$ f (x + i, y + j ) 
\\\\
To check if same result, the kernel before and after 180 degrees are same
\\\\
Intuition of Sobel filter\\
gx (x, y) ≈ f (x + 1, y) − f (x - 1, y) \\
correlate [−1, 0, 1] with the image\\
$[-1,0 ,1]$\\
$[-2 ,0 ,2]$ = $2[-1, 0 ,1]$ (put more weights at center)\\
$[-1 ,0, -1]$\\
\\
Combine the both result using the horizontal and vertical Sorbel to get the final edge image
\\
3x3 kernel is used as we need to include the neighbours as single row or single column kernel is sensitive to noise
\\
\\
\\
\subsection*{Derivatives}\\
does not matter the order of the derivative and gaussian blur\\
gaussian filter to remove noise before laplacian applied\\
\\\\
vertical shows pixel of image row\\
horizontal shows the gradient value \\
\\
f(x) grey level value\\
f'(x) gradient value\\
$[1,-2,1]$\\
f''(x) gradient of the gradient \\
\\\\
gxx (x, y) $\approx$  f (x - 1, y) − 2 \cdot f (x, y) + f (x + 1, y)\\
gyy (x, y) $\approx$  f (x, y - 1) − 2 \cdot f (x, y) + f (x, y + 1)
\\
\subsection*{First order derivative}\\
Sobel is a combination of derivative kernel\\
Sobel results in wide edges as it is first order\\
First order derivative (thicker edges)\\
Roberts, Prewitt, Sobel filter
\\
\subsection*{Second order derivative}\\
Second order derivative can know the exact edge and detect 1 px thin edge
Smoothing is needed\\
DoG (difference in gaussian)\\
laplacian is the second order derivative (most sensitive to noise)\\
-1 white 0 grey 1 black
\\
\\
To approximate the 2nd order derivatives\\
gxx (x, y)  $\approx$ f (x − 1, y) − 2 · f (x, y) + f (x + 1, y) 
represents [1;-2;1]
\\
gyy (x, y)  $\approx$ f (x, y − 1) − 2 · f (x, y) + f (x, y + 1)
represents [1 -2 1]
\\\\
\\\\
\subsection*{Hysterisis}\\
pixels below some value 0\\
hysterisis join the strong and weak pixels\\
thresholding depends on the image\\
\\
cv2.canny is interpolated version
\\
random forest is a learned model from human markups\\
\\gradient shift dark to light and light to dark
\\\\
is the center a local maximum or not
if yes keep it as part of the edge else discard it
\\
%don’t know how far this analogy is going to uphold
\\\\
\
\\\\
Q: would there be multiple local maxes after performing the non-maximum suppression?\\
A: no multiple local maxes
\\\\
Q: is horizontal gradient detection always from left to right? similar for vertical case.\\
A: left to right\\
\\
Q: Could you explain the intuition behind why cross correlation isn't commutative/associative, even tho convolution is just flipping the kernel?
\\
A: [1, 2, 3] cross correlation [3, 2, 1] is not the same as [3, 2, 1] cross correlation [1, 2, 3]
\\\\
$[1, 2, 3]$ cross convolution [0, 1, 0] is [3, 2, 1] so it is not identity
\\
\\\\\\
Q: From slide 9, how is the 1D derivative filter constructed from the finite differences discrete version equation with h = 2?\\
A: 2 elements we are using 
\\\\
Q: When convolution/cross-correlation is mentioned, is full padding assumed by default?\\
A: is not full padding, lecture examples are only full padding
\\\\
\noindent
Q: is padding to just inserting zero rows and zero columns or that plus blurring/interpolation? It seems to have been used both ways
having to do nn interpolation (bilinear, linearly solve between 2 neighbours)0 interspersed in rows and columns
\\
%blue line falls can estimate what the ratio is then multiply by pi
blurring to reduce noise (to avoid enhancing the noise)
\\\\
Q: I think it was mentioned last lecture that convolutions are generally invertible. How do we reconcile this with the notion that blurring is lossy?\\
A: convolution is multiplicity in the inverse domain (do division can have problems)\\\\
blurring in itself is not lossy
only downsampling
\\
for every single elements, composed on several pixels, mixture of weights, same kernel applied to next kernel
blurring + downsampling makes it lossy
\\\\
factor all of this in the system of equations should be able to recover the image
\\\\
Q: why not take the residual of the original versus upsampled of previous image\\
A: remove some values these are the values we want to preserve in the residual
else will keep some of the detailing 
\\
\section*{L6 notes}
\subsection*{Some terms}
bandwidth is too narrow not very representative
\\
we don't know what is the correct/ incorrect value for bandwidth
\\
sum of all the gaussian forms the new curve
\\
dirac delta - bandwidth is very small
\\
number of textons can be smaller/larger than the filter banks
\\
euclidean distance between histogram ends up compensating for each other
\\
What are textures used for?\\
cue to tell us on the underlying 3d structure
\\
scale of the envelope, single or many wavelengths\\
\\
running across all the images where each image is a pixel
\\run the cluster where every single pixel and image is a datapoint
replace all pixels in the email with the texton id, compute the histogram based on the samples, each of the texton image is represented with a histogram
\\
texton id is per pixel
\\
x and y then the texton would not be purely texture
\\\\
texture is independent feature to the segment it belongs to, we don't have to do segmentation to find the segment in the first place
\\
the responses is for the whole filter bank
\\
Do k-means twice\\
first k means to form texton dictionary\\
second k means NN search (can also apply mean-shift)\\
\\
\\histogram is 100 dimension vector also
\\
Feature Representation\\
Filter bank Response\\
Texton histogram\\
\\
Note: Texton histogram as a feature for segmentation will not give a clear segmentation, problematic edges, results in a pixelated image for the texture (not localised)\\
Good localization, pixel from the 1 pixel or all over is the same
\\
CNN is aggregation of many filters, take a feature response and apply another kernel to it
\\
blurring each channel independently (2d blurring),
now we are blurring the filter response
\\
we have consider the location of the pixel for segmentation
\\
region is bounded by two separate histograms, then u would have separate results
\\
Q: Are we able to somehow use a filter to extract coloraturas as a filter inside the filter bank? So that we can run clustering algo on both texture and colour at the same time?\\
A: you can mix the colors if you want, create a gabor filter for each channel\\
\\
Q: can I clarify whether textons are generated based on clustering done over all filter results within a filterbank? Or is it done using only some sort of 'most differentiating' filter result?\\
A: filter bank only to apply to all images
\\
apply 2d convolution
\\
Q: how does using the same filter bank ensure the bins are the same\\
A: having very different feature response\\
1 image per type then maybe
\\
Q: Why 3 clusters?\\
A: there is nothing to distinguish the skin from the background
\\
\subsection*{Quiz 1}
Red and green considered equally, but blue is not considered at all\\
Maximum shades of grey will not be the shade
\\
\section*{L7 notes}
\subsection*{Some terms}
SSD\\
keypoints are also corners or interest points\\
low error will look dark, high error will look bright\\
shape is going to affect the ellipse, size does not matters	\\
H = [0 0 ; 0 C] gradient is x direction is 0\\
H = [A 0 ; 0 0] gradient is y direction is 0\\
\\
greedy approach non-maximum suppression has keypoints approximately in the same location as it looks for areas that has high contrast
\\
automatic scale selection, harris is equivariant to changes of scale\\
\subsection*{Weights of derivative}
gaussian before we do the summation for the weights, more weight at the center, less weights at the edge
\\
\\
Instead of applying the effects, we are looking at two viewpoints under two different lighting conditions
\\\\
highest response has a signal has characteristic scale that has the same as gaussian, width of signal corresponds in signal charasteristic
	\\\\
	Harris operator is more efficient compared to eigen value decomposition\\
	If the region is unusual then it is difficult for matching, approximated distinctiveness with SSD error
	\\
	Linearize with 2nd moment matrix\\
	Quantify distinctiveness with eigenvalues of H 
	\\
	R is the cornerness is defined as \\
	R = det(H) - k $trace^{2}$(H)
	\\\\
	Values can be obtained from the matrix itself
	\\\\
LoG finds blobs (keypoint detector that tries to find roughly circular regions)
maximum scale is already built-in
(maximum size of the kernel wrt to image)
\\\\
Auto correlation: Create a template of the patch, shift it around where I use the local patch
\\
Q: Is template the benchmark for comparison?
A: strongest peak where it matches to itself
Refer to template matching\\
\\
No matter where u shift the template, will get a strong response at the sky\\
Third dimension is the intensity
\\
We need 3d… How do we analyse the points without looking at 3d\\
\\
$\lambda_{min}$ and $\lambda_{max}$  are perpendicular to each other\\
\\
SSD error E(u, v)\\
Error will be 0 or greater\\
H is always positive definite\\
Semi definite where there is 0 case, flat region only in synthetic images
\\
Cross section is approximately circular ( a lot of change in all directions)\\
Direction of the fastest change perpendicular to the slowest change will be approximately equal\\
Which direction I change in it is going to be different
\\
Contrast causes the threshold\\
$\lambda_{min}$ would be small as $({\lambda_{min}})^{-\frac{1}{2}}$
where there is no change on the straight line \\
\\
\\
Q: Clustering the responses of the textons\\
A: The response are the data points, each feature response is 1 dimension, each data point cluster in 38 dimensions
\\
Dimensionality: Refer to L6 supplements\\
One blue dot is 1 pixel, x location is 1 feature value and y location is 1 feature value
\\
Eigenvalue decomposition of pixel then u would get every pixel also\\
axis length is very long
\\\\
Q: Texture and template matching are they the same?\\
A: Correlate the long, what is the pattern vs the individual image. Detecting many texture in 1 image, so the feature response will be different in different part of images. Only a particular filter is tuned to that signature response. Normally we don't have to hand-create the filters. Deep learning to learn whole bunch of features, or gabor filter, can tune the filters. Over many dimensions can get a unique response to that example.
\\\\
Q: Filter bank to find the size\\
A: Large enough filter bank or some way to count 
\\\\
Binary classification if the sheet is defect or not 
\\\\
Q: Why is the skin and background cannot be segmented?\\
A: Texture of the background is very large, approximate scale of the marble is on par with the arm. Doing agnostic segmentation, we don't model the content in the image, only based on filter response. There is no reason why it should end up in certain segmentations.
\\
Clustering is done over all of the pixels over all of the images. Creating the local histogram is trial and error
\section*{L8 notes}
\subsection*{Some terms}
SIFT - scaling iterative feature response\\
Pixel wise difference (gradients) - more sensitive to relative location but sensitive to deformations
\\
Color histogram - 3d histogram \\
What size of the cells do we want, fine grids, overlap etc
Bin sizes - more finely tuned, a small shift will affect the histogram
\\
Compare histogram - chi square
\\
Orientation normalization\\
\subsection*{Averaging}\\
Bimodal distribution average end up with dominant orientation with no gradient\\\\
MOPS or GIST is no longer used as a descriptor\\
\\\\
\subsection*{Mode - SIFT}\\
SIFT - understand the implication of doing this\\
Can be combined with other keypoint detector
\\
as the threshold is larger, it removes more keypoints, depends on illumination difference\\
\\
poorly localised if it sits on an edge and not a corner\\
corners have high curvature, edges have low curvature\\
\\
needs to be clear where the keypoint is matched to when moving in a straight line\\
\\
ratio of eigenvalues to be below the edge thresh
high threshold more tolerant of edges\\
smaller bar to be corner, longer bar is edge\\
white on black and black on white difference\\
black bar cannot be keypoints as we also have the surrounding neighbors\\
\\
We don't assign gradients in original orientation - dominant orientation
\\
Taking the original histogram with some wrapping operation, form of normalisation wrt dominant orientation\\
Normalize to unit length\\
If it exceeds the threshold set the threshold value, rectify and assign it. Then renormalize.\\
Clamping is to encourage invariance to small changes to illumination, over count in certain orientations.
\\
can use in the dense form where it can use in any rotation \\
High robust
\\
\\
\subsection*{Feature matching}
\subsection*{Image transformation}
Filtering\\
Only changes on the actual image intensities\\
G(x) = h \{ F(x)\ \}
Warping\\
G(x) = F \{ h(x)\ \}
changes the location of the pixels
\\\\
\section*{L9 notes}
\textbf{2d projective transformation}\\
affine: parallel lines are still preserved
projective: also known as a homography, try to solve for the parameters for  a projective transformation\\
Not an arbitrary matrix with 9 degrees of freedom (3x3)\\
Solve for H using direct linear transform
\\
Apply homography to align two images together
\subsection*{SVD}
$A = U \sum V^{T}$\\
A - 8 $\times$ 9\\
U - 8 $\times$ 9\\
$\sigma$ - 9 $\times$ 9\\
V - 9 $\times$ 9\\
\\
V is a singular matrix\\\\
singular values found on diagonal values of sigma\\
columns of V are single vectors
\\
N number of correspondences
\\ 
Normalization such that the centroid is (0, 0) of the similarity transformation is a whitening step
\\
RANSAC deal with outliers, sample from the set of noisy observations
\\
Even though we can sample more, we usually s to be equal to the number of the outliers
\section*{L10 notes}
\subsection*{Optical flow}
The bigger the arrow, the stronger the flow
\\
Small motion assumption - Linear wrt to the location
\\\\
Constant flow\\
All flow in the same region will have a flow
\\
Ax = b A is not a square matrix
\\\\
Have different motions to avoid aperture problem\\
motion is not small, we end up with wrong matches and estimate shift are actually not smaller than the actuals shift
\\
$\rightarrow$ reduce the resolution
\\
\\
smoothing the flow vectors are 
\\
warp and upsample,
compute the flow until we reach the highest resolution
\\
Line and smooth location, we cannot take the inverse, can only get sparse motion flow with Lucas kanade
\\Suitable for large images
\\\\
Horn Schunck\\
Assumption: smooth flow field\\
As a min problem, u and v to be small \\
\\
Each pixel flow can be different
\\
Difference in flow between adjacent pixels are small
\\
uniformly flow field will have the lowest flow field compared to the random flow field
\\
We also consider the right and top neighbour so we won't have to double count everything
\\
$u_{i, j}$ appears 4 times as it is being referenced by the neighbours
\\
Only suitable when in the video frames movements are small
\\
Does not mean that there is motion in the underlying object\\
For every model, go back and question our assumptions\\
\\\\
Motions are large how to solve this?\\
Coarse to fine estimation using LK
\\
Or non linear assumptions like Horn Schuncks
\\
\section*{L11 notes}
\subsection*{Tracking}
Tracking the speed of the action\\
Estimate volume of heart chamber\\
We can't rely only on optimal flow for tracking\\
\\
First search at coarser scale\\
Only need to search on local region based on previous state
\\
Based on initial location so it has to be good
\\
Can have other types of transformation, rather than just homography (projective transform)
\\
no closed form to express I as it has no parameters for p	\\
Can only compare as much pixels as in the template
\\
Do in a matrix instead of a loop form, each element is independent so we can parallelise
\\\\
\textbf{KLT algorithm}\\
Refresh population of possible corners as features of tracking\\\\
\textbf{Duality of feature tracking vs Optimal flow}\\
Feature tracking\\
Follow specific pixels
As we solve optical flow problem then we can track features, vice versa\\
\subsection*{Mean shift}
Initialize for a specific data point and apply to every point\\
Every point that end up at the same attraction basin belongs to same centroid\\
Move the points to the centroid to calculate the mean shift vector for every single point\\
Location that we end up at are the local maxima called attraction basin\\\\
The reason we do for every single point for segmentation is we want to identify the membership for each mode
\\
\\
b is the binning function\\
Look at m which bin does it go into, if it go to mf bin, then $b(x_{n} - m) $= 0 and delta function goes to 1\\
Contribution to q is 0, does not contribute to descriptor value\\
We are trying to pick out the points that have the colors\\
Normalizing with the size of the target\\
\\
Bhattacharyya co-efficient, L2 is not very ideal for mean-shift\\
Find the peak of mode
or maximizing over all the possible locations y
\\
Assign to new target and calculate the candidate again\\\\
\textbf{Challenges of mean shift}\\
We assume the bounding box is going to move the same but if the person is moving the hand further from the camera, so it is not a good assumption
\\\\
Repeating the procedure from frame to frame, we may suffer issue from drift (cannot count each color pixel equally), might need to reinitialise at some point\\\\
\\
Target may leave out of the frame, there are a lot of objects in the scene\\
Stationary vs moving cameras\\
ID shift, move to another object\\
Moving background foreground and background are changing 
\\
\section*{L12 notes}
\subsection*{Data driven image classification}
Learning classifiers based on data\\
Train classifiers using KNN or DNN\\\\
\textbf{Challenges in CV}\\
Invariance and generalization
\\\\
\textbf{Where deep learning comes in}\\
Every single image in orientation and shape\\
Extrapolate some form of semantics\\
\\
\textbf{AI}\\
Rule based system that uses expert indulge human intelligence
\\
Narrow AI only do a dedicated task\\
Small sliver of human intelligence
\\
Neurons are non linear processing unit
\\
NN is not close to how the brain even works
\\
Mainly done for statistical pattern recognition
\\\\
Non linearity (Perceptron) are building blocks, weighting the sum which passes through a non-linear activation function
\\
Features are formed with cascade of functions that transform features
\\\\
\textbf{Perceptron Mark I}\\
Original perceptron was created to do image processing\\
Each photocell is sensitive to light\\
Wires are creating different combinations of inputs\\\\
Fully connected - Every single neuron has a connection to the output layer, weighted summation of every single pixel, each representing a probability of the class\\
\\
Locally connected instead of fully connected
\\
Flatten the matrix into a vector, each pixel is 1 element in the vector which correspond to the input	
\\
Hidden units = input\_width $\times$ input\_height
\\
Parameters for 1 hidden unit = Hidden unit ^2\) (+1 is negligible)
\\\)\\
\textbf{Stationarity}\\
Definition of corners does not change\\
Relative comparison to the local neighbour hood upper left and bottom right\\
Apply same weights to to all neurons colored different
\\
Otherwise we would need different weights\\
Every neuron is centered at every pixel\\
Square area is called the receptive field of the perceptron\\
If it has a reduction in half the dimension, receptive field at 3 $\times$ 3 will correspond to 6 $\times$ 6 in the original image\\
Will progressively increase even if kernel stay in same size due to reduce resolution
\\\\
Perceptrons are using convolution operation - sweep the kernel over image 
\\
\\
Kernels whose weights are learned\\
Filtering extract edges, orientations of edges
\\
E.g. Gabor, Sobel filter
\\
Instead learn weights for the filters
\\\\
\textbf{Feature maps}\\
Also called a channel\\
\\
NN also uses pooling rather than simply convolution
\\
After pooling the resolution would be very coarse\\
Aggregate response into a simple value like downsampling
\\
ReLu rectified to 0
\\
Convolution is done by many kernels rather than just 1, using parameter sharing
\\
\\
ImageNet is 3 orders of magnitude larger than previous Nets
\\
Top 1 = 100 - 1\\
Top 5 = As long as u have 5 of the classes in the top of the probabilities, it is also considered correct, top 5 most confident correspond to ground truth
\\
\subsection*{Extended Image classification}
Not only find out what it is, also where it is
\end{document}
