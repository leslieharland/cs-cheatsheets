\documentclass[11pt]{article}
\usepackage{listings}
\begin{document}
\noindent
\section*{L3 notes}
\subsection*{Some terms}
d(h * f) h convolve with f \\
Know what the math equations are doing and apply them\\
%There is no blue cell after green cell
Pad the sides with 0 and shift the kernel across the numbers\\
floor(kernel size / 2)\\\\
$[1,1,1 ]$ symmetric\\
Linear (shift invariant) = output Is scaled accordingly\\
\\
\
Correlation filter to shift pixels in image
\\\\
Normalised = summed up add to 1\\
Effects are negligible for large images\\\\
Gaussian kernel depends more on the sigma\\\\
Salt and pepper = isolated pixels that are problematic\\
Smooth away the impulse and spread it all around\\
Impulse is salt without the pepper\\\\
Higher signal amplify both the signals and noise
\\\\
Histogram stretching adjust contrast\\\\
\
Bitmap save the pixels independently\\
\subsection*{Laplacian}
the process to get the residual is [upsample (fill with 0s) $\rightarrow$ blur $\rightarrow$ subtract from original]\\
(the reason for the blur is to "spread out" the values over the filled-in black pixels)
\newpage
\noindent
Q: Does spatial quantization mean the quantisation of values of x and y? i.e. x and y can only take discrete integer values?\\
A: spatial quantization basically happens because we cant represent continuous x and y coordinates in a discrete system\\
\\
Q: It seems that you always regard intensity as integers, is it a routine for this course?\\
A: Yes and no\\
\\
Q: does adjusting abstract and brightness simply refer to the linear mapping or also include the non-linear gamma mapping\\
A: Non-linear mapping also just the contrast\\
Photography operation we don’t use stretching ( we use images that stretch over the entire range)
\\\\
Q: In practice, is convolution used more than correlation?\\
A: Convolution\\
\\
Q: Is reconstructing the original image using laplacian pyramid 100\% lossless?\\
A: Up to the difference of quantisation
\\\\
Q: Can we compensate motion blur by finding the motion blur kernel and use the inverse kernel to demotion blur?\\
A: Yes, finding the kernel is always hard
\\\\
Q: Is convolution invertible? In a sense that let's say we have a “convolved” image, is there always a filter such that we can recover the original image using convolution? e,g, undoing blur convolution using a specific sharpening convolution\\
A: If we throw away half of the pixels
\\\\
Q: Does laplacian pyramids take 2* space than gaussian pyramids simply because laplacian pyramids have 2 sets of the pyramids (gaussian + residual)?\\
A: Same amount of space
\\\\
%Low pass filtering signal into low domain frequency
%High frequency represent sharp edges
%Slow changing parts to remain and we take care of the fast changing pixels
%We can take it as a blurring filter




%2k + 1 is the size of the kernel

\section*{L4 notes}
\subsection*{Some terms}
mean filter does not remove all noise and blurs the image
only isolated values 0-255
\\\\
median filter remove all noise and blurs the image slightly
arrange in increasing starting from 0
\\\\
difference filter sum up all the values in row 
same value for the entire row
\\\\
each pixel has a tangent plane
\\\\
kernel size controls the strength of the filter
\\\\
border problem output is reduced in size for the pixels along the edge
\\\\
 the image patch and the template always contain positive numbers, cos $\Theta \in [0, 1]$, i.e., the output of normalized cross-correlation is normalized to the interval [0,1], where 0 means no similarity and 1 means a complete similarity. 
\\\\
region of interest can benefit template matching
\\\\
image has discrete representation we need approximation, positive gradient value when the image change from dark to bright and negative when reversed image
\\\\
Image sharpening g(x,y)=f(x,y)- f(x,y) $\circ $ h(x,y)
\\\\
Magnitude = $\sqrt{(gx2 + gy2 )}$\\\\
Approximated magnitude = $|$gx $|$ + $|$gy $|
$
\\
non maxim suppression
if edge has a magnitude too small connected to another pixel above a threshold, don’t prune 
\\\\
extract edges to detect start and end edges , useful for 3d to know the shape and geometry\\
why? resilient and lighting and color useful for recognition
\\
\\
\subsection*{Template matching }
This object is now the template (kernel) and by correlating an image with this template, the output image indicates where the object is. Each pixel in the output image now holds a value, which states the similarity between the template and an image patch (with the same size as the template) centered at this particular pixel position. The brighter a value, the higher the similarity.
\\\\
Purely correlation since strongest response is when original image exactly matches output\\
what happens when u zoom the baby's face 2x (refer to image)?
\\\\
Template is not symmetrical
\\
\subsection*{Neighborhood processing }\\
Neighbor pixels play a role when determining the output value of a pixel 
\\
\subsection*{Convolution vs Correlation }\\
Convolution (rotated 180 degrees)\\
h(i, j ) $\cdot$ f (x - i, y - j ) 
\\\\
Correlation’\\
h(i, j ) $\cdot$ f (x + i, y + j ) 
\\\\
To check if same result, the kernel before and after 180 degrees are same
\\\\
Intuition of Sobel filter\\
gx (x, y) ≈ f (x + 1, y) − f (x - 1, y) \\
correlate [−1, 0, 1] with the image\\
$[-1,0 ,1]$\\
$[-2 ,0 ,2]$ = $2[-1, 0 ,1]$ (put more weights at center)\\
$[-1 ,0, -1]$\\
\\
Combine the both result using the horizontal and vertical Sorbel to get the final edge image
\\
3x3 kernel is used as we need to include the neighbours as single row or single column kernel is sensitive to noise
\\
\\
\\
\subsection*{Derivatives}\\
does not matter the order of the derivative and gaussian blur\\
gaussian filter to remove noise before laplacian applied\\
\\\\
vertical shows pixel of image row\\
horizontal shows the gradient value \\
\\
f(x) grey level value\\
f'(x) gradient value\\
$[1,-2,1]$\\
f''(x) gradient of the gradient \\
\\\\
gxx (x, y) $\approx$  f (x - 1, y) − 2 \cdot f (x, y) + f (x + 1, y)\\
gyy (x, y) $\approx$  f (x, y - 1) − 2 \cdot f (x, y) + f (x, y + 1)
\\
\subsection*{First order derivative}\\
Sobel is a combination of derivative kernel\\
Sobel results in wide edges as it is first order\\
First order derivative (thicker edges)\\
Roberts, Prewitt, Sobel filter
\\
\subsection*{Second order derivative}\\
Second order derivative can know the exact edge and detect 1 px thin edge
Smoothing is needed\\
DoG (difference in gaussian)\\
laplacian is the second order derivative (most sensitive to noise)\\
-1 white 0 grey 1 black
\\
\\
To approximate the 2nd order derivatives\\
gxx (x, y)  $\approx$ f (x − 1, y) − 2 · f (x, y) + f (x + 1, y) 
represents [1;-2;1]
\\
gyy (x, y)  $\approx$ f (x, y − 1) − 2 · f (x, y) + f (x, y + 1)
represents [1 -2 1]
\\\\
\\\\
\subsection*{Hysterisis}\\
pixels below some value 0\\
hysterisis join the strong and weak pixels\\
thresholding depends on the image\\
\\
cv2.canny is interpolated version
\\
random forest is a learned model from human markups\\
\\gradient shift dark to light and light to dark
\\\\
is the center a local maximum or not
if yes keep it as part of the edge else discard it
\\
%don’t know how far this analogy is going to uphold
\\\\
\
\\\\
Q: would there be multiple local maxes after performing the non-maximum suppression?\\
A: no multiple local maxes
\\\\
Q: is horizontal gradient detection always from left to right? similar for vertical case.\\
A: left to right\\
\\
Q: Could you explain the intuition behind why cross correlation isn't commutative/associative, even tho convolution is just flipping the kernel?
\\
A: [1, 2, 3] cross correlation [3, 2, 1] is not the same as [3, 2, 1] cross correlation [1, 2, 3]
\\\\
$[1, 2, 3]$ cross convolution [0, 1, 0] is [3, 2, 1] so it is not identity
\\
\\\\\\
Q: From slide 9, how is the 1D derivative filter constructed from the finite differences discrete version equation with h = 2?\\
A: 2 elements we are using 
\\\\
Q: When convolution/cross-correlation is mentioned, is full padding assumed by default?\\
A: is not full padding, lecture examples are only full padding
\\\\
\noindent
Q: is padding to just inserting zero rows and zero columns or that plus blurring/interpolation? It seems to have been used both ways
having to do nn interpolation (bilinear, linearly solve between 2 neighbours)0 interspersed in rows and columns
\\
%blue line falls can estimate what the ratio is then multiply by pi
blurring to reduce noise (to avoid enhancing the noise)
\\\\
Q: I think it was mentioned last lecture that convolutions are generally invertible. How do we reconcile this with the notion that blurring is lossy?\\
A: convolution is multiplicity in the inverse domain (do division can have problems)\\\\
blurring in itself is not lossy
only downsampling
\\
for every single elements, composed on several pixels, mixture of weights, same kernel applied to next kernel
blurring + downsampling makes it lossy
\\\\
factor all of this in the system of equations should be able to recover the image
\\\\
Q: why not take the residual of the original versus upsampled of previous image\\
A: remove some values these are the values we want to preserve in the residual
else will keep some of the detailing 
\\
\section*{L6 notes}
\subsection*{Some terms}
bandwidth is too narrow not very representative
\\
we don't know what is the correct/ incorrect value for bandwidth
\\
sum of all the gaussian forms the new curve
\\
dirac delta - bandwidth is very small
\\
number of textons can be smaller/larger than the filter banks
\\
euclidean distance between histogram ends up compensating for each other
\\
What are textures used for?\\
cue to tell us on the underlying 3d structure
\\
scale of the envelope, single or many wavelengths\\
\\
running across all the images where each image is a pixel
\\run the cluster where every single pixel and image is a datapoint
replace all pixels in the email with the texton id, compute the histogram based on the samples, each of the texton image is represented with a histogram
\\
texton id is per pixel
\\
x and y then the texton would not be purely texture
\\\\
texture is independent feature to the segment it belongs to, we don't have to do segmentation to find the segment in the first place
\\
the responses is for the whole filter bank
\\
Do k-means twice\\
first k means to form texton dictionary\\
second k means NN search (can also apply mean-shift)\\
\\
\\histogram is 100 dimension vector also
\\
texton histogram as a feature for segmentation will not give a clear segmentation, problematic edges, results in a pixelated image for the texture
\\
blurring each channel independently (2d blurring),
now we are blurring the filter response
\\
we have consider the location of the pixel for segmentation
\\
region is bounded by two separate histograms, then u would have separate results
\\
Q: Are we able to somehow use a filter to extract coloraturas as a filter inside the filter bank? So that we can run clustering algo on both texture and colour at the same time?\\
A: you can mix the colors if you want, create a gabor filter for each channel\\
\\
Q: can I clarify whether textons are generated based on clustering done over all filter results within a filterbank? Or is it done using only some sort of 'most differentiating' filter result?\\
A: filter bank only to apply to all images
\\
apply 2d convolution
\\
Q: how does using the same filter bank ensure the bins are the same\\
A: having very different feature response\\
1 image per type then maybe
\\
Q: Why 3 clusters?\\
A: there is nothing to distinguish the skin from the background
\\

\section*{L7 notes}
\subsection*{Some terms}
SSD\\
low error will look dark, high error will look bright\\
shape is going to affect the elipse, size does not matters	\\
H = [0 0 ; 0 C] gradient is x direction is 0\\
H = [A 0 ; 0 0] gradient is y direction is 0\\
\\
greedy approach non-maximum suppression has keypoints approximately in the same location as it looks for areas that has high contrast
\\
\subsection*{Weights of derivative}
gaussian before we do the summation for the weights, more weight at the center, less weights at the edge
\\
\\
Instead of applying the effects, we are looking at two viewpoints under two different lighting conditions
\\\\
highest response has a signal has characteristic scale that has the same as gaussian, width of signal corresponds in signal charasteristic
	\\\\
	Harris operator is more efficient compared to eigen value decomposition\\
	Values can be obtained from the matrix itself
	\\\\
LoG finds blobs (keypoint detector that tries to find roughly circular regions)
maximum scale is already built-in
(maximum size of the kernel wrt to image)
\end{document}
