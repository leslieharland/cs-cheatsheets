\documentclass[11pt]{article}

\begin{document}
\noindent
\subsection*{Some terms}
Model-free: not learning the transition and reward, only need to get observations from the real world or simulator\\
Prediction to control - prediction into a control policy\\
\\\\
\subsection*{Monte Carlo}
$\pi_{UCT}(n) = arg max_{a} \big(Q(n, A) + c \sqrt{\frac{log_2{(N(n))}}{N(n, a)}}})$\\
V(A) = $\frac{\# A | A = 1}{\# A + \# B}$
\subsection*{Generalized policy iteration}\\
Take argmax get the policy\\
Run the policy and estimate a new Q function\\
Don't have to evaluate the policy fully\\
Do one step and update one data (Q function)\\
Update greedily wrt the Q function\\
By being greedy (get a new policy) and improve it
\\\\
Monte Carlo\\
Take many trajectories and average the trajectories
\\
Temporal difference learning\\
Immediately update our prediction just after one step\\
But there is no return from the long trajectory\\
Current return of u\\
U is the estimate
\\
Monte Carlo error: considers all the previous steps
\\
vs.\\
TD error: how difference are my difference at this time step
\\
\subsection*{n step TD}
better tradeoff between bias and variance
\\
When n is $\infty$, it becomes Monte Carlo\\
Average can somewhat reduce variance (need less data to get good estimate, learn faster as it converges faster)\\
\\
TD($\lambda$) \\
n step is weighted by $\frac{\lambda}{n-1}$\\
0 forget the future take 1 step\\
1 close to 1 important for a long time
\\
We are averaging the estimates (move expensive)
\\
\subsection*{ADP}
Run vi or pi\\
not trying to estimate the model only the values
\\
need to have simulator of the real world to go on with RL
\\
Build a model using function approximation\\
What is the transition function to use?\\
What reward to use\\
\\
Successful has been parameterised Q function, supervised learning etc.
\\
Model-free are easier to work with\\
\\
We usually mix MC with TD in practice, less estimate and not much variance\\
\\
\subsection*{TD}
Assumption is it is always markovian in real RL tasks in the environment\\
TD learning applies assumption to estimate\\
Initializer can be 0\\
A, 0, B, 0\\
Markovian assumption, if it sees A and B then it gives the reward of 0.
Does not give the right values of the states because it makes an assumption that is not true of the environment\\
If it is true, then it will converge to  correct value functions\\
Update equation\\
 V^{\pi}(s) \leftarrow V^{\pi}(s) + \alpha(R(s) + \gamma\ V^{\pi}(s') - V^{\pi}(s))
 \\
 1 - \alpha can be obtain by extracting - V^{\pi}(s) out and simplifying
 1\ -\ $\alpha$ can be obtain by extracting - $V^{\pi}(s)$ out and simplifying
\subsection*{SARSA}
TD control\\
Uses GPI, on policy (improve the same policy)\\
target = R(s) + $\gamma$ Q(s', a')\\
SARSA will converge to optimal policy on tabular cases\\
\\
MC on the windy gridworld\\
Converges slower but eventually reaches optimal\\
Have to use old policy until end of the trajectory
\\
SARSA changes policy at every time step, may wander around at the start, but will continue learning
\\\\
$\epsilon$ greedy: Take the action suggested by the Q function or random exploration
\\
Improving the current policy with $\epsilon$ greedy, when present does not converging to the optimal policy
\\
trying to improve the policy\\
environment might change, world isn't quite a MDP, it can be robust, model may change over time \\
Might still fall off the cliff
\\
Predictive problem learning (passive TD) has some restrictions 
\\
Update only at next state
\subsection*{Q-learning}
Only have the q function (acts as policy on its own)\\
Not in SARSA: $\gamma$  max a \\
best action at the next action of Q function\\
\\
E.g currently have a deployed policy, at wild change the policy with SARSA, train the target policy\\
Behavior policy can be some other policy that is running, generate data to train the policy
\\
If stop learning then Q-learning will outperform Sarsa
\\
\\
Similar concept to $\epsilon$ greedy\\
Let's go to part of the space that is exploit less\\
Issues when state space is large\\
\\
Update policy then take argmax - given the current policy from the Q function, estimate a new Q function and construct a new policy by taking argmax of Q function\\\\
Q learning can learn off policy (doesn't follow the policy given, uses different policy)\\
Take the max over the next state\\
Minimizing the difference between the target and the current estimate\\
If the count is small give bonus, if we try many times reduce bonus\\
\\
Why don't u update all the Q at each state?\\
Update is only done for all the state-action\\
Not the final Q-table, it will interact with the environment and update the table
\\
What if the transitions are non-deterministic?\\
probability to move to a state\\
Even it is probability, it will move to that state, no longer a probability so take Q value of that state, expected Q learning
\subsection*{Function approximation}
Number of states are huge, huge state space\\
Approximate (estimate) Q value with function approximator\\
\\
Figure out what are the important features and extract out to define each state
E.g. for a chess, queen is worth 5 points, pawn worth, knight worth\\
RL will learn the parameter vector $\Theta$
\\
Allows generalisation observations so that it can work on entire state space\\
Small fraction of the state\\
Applications: Go, Deep neural network, OpenAI5
\\
MCT Search on simulator, for online search value function and combine with MCTS
\\
Do learning with function approximation\\
Runs in the form of ridge regression\\\\\\
\textbf{Online learning}\\
One example, learn an example and update the weight vector\\
Update theta to improve estimate, step in the direction of the negative gradient
\\
$\hat{U}(s)$ function approximator, estimate from the function\\
target $u_{j}(s)$At every time step it is the sum of the reward to the future, measured from the environment\\\\
It could be the q function\\
Semi gradient - don't include in the computation\\
gradient wrt the parameter $\Theta$\\
we treat $\gamma\hat{U}(s')$ as a fixed target
\\
When we move on-policy we lose most of our policy, have to do lots of tuning e.g. doesn't converge
\\
Deep Q-learning\\\\
Techniques to do better: Experience replay\\
large buffer D, old sample does not mix with the recent one\\\\
$(s_t, a_t, r_{t+1}, s_{t+1})$ max over a without the k\\
learn and freeze it and use as target $\theta^{-}$ \\
$\theta^{-}$ Q function from current - c steps\\\\
High profile achievement as one architecture can work across 50 atari games\\
Set a new target and change every c steps\\\\\\
Extract with a linear function of features\\
E.g. position of the ball
\\
Need two frames to determine the direction we are going\\
provably good - how long would it take to go to epsilon
\subsection*{Policy search}
Mapping from state to action\\
Drop the assumption that policy is of Q(s, a)\\
Deep learning is continuous\\
Max function is not differentiable everywhere\\
max function isn't linear\\
Gradient descent can differentiate that
\\
Actions are discrete can't find gradient wrt a
\\
\end{document}
